\documentclass{article}

\usepackage[utf8]{inputenc}
\usepackage[T1]{fontenc}
\usepackage{hyperref}
\usepackage{tabularx}
\usepackage{array}
\usepackage{fancyhdr}
\usepackage{graphicx}
\usepackage[a4paper]{geometry}
\usepackage{multicol}
\usepackage{listings}
\usepackage{tabto}

\title{Projet de BigData \\ Les bases NoSQL}
\author{par Jordan Baudin, Corentin LeGuen et Geoffrey Spaur}
\date{11 janvier 2018}
\pagestyle{fancy}
\lhead{Projet de BigData - Les bases NoSQL \\ \textbf{M2GIL} - Jordan Baudin, Corentin LeGuen et Geoffrey Spaur}
\rhead{\includegraphics[scale=0.5]{logo_univ_rouen.png}}
\setlength{\headsep}{1cm}
\begin{document}

\maketitle
\newpage
\tableofcontents{}
\newpage
\section{Présentation}
  \paragraph{}
  Ce projet a pour but de comparer différentes bases de données NoSQL.
  Nous allons prendre les bases données CouchBase et MongoDB.
  
  
\section{La base de donnée CouchBase}
\subsection{Les modalités d’installation}
  \paragraph{} TODO
\subsection{Les méthodes d’insertion de données}
  \paragraph{} TODO
\subsection{Le langage de recherche}
  \paragraph{} TODO
\subsection{L’indexation interne}
  \paragraph{} TODO
\subsection{Le support de la concurrence d’accès}
  \paragraph{} TODO
\subsection{L’architecture du système}
  \paragraph{} TODO
\subsection{Les techniques de distribution}
  \paragraph{} TODO
  
  
  
  
\section{La base de donnée MongoDB}
\subsection{Les modalités d’installation}
  \paragraph{Téléchargement} Pour l'installation de MongoDB, il vous suffira
  dans un premier temps de \href{https://www.mongodb.com/download-center#community}{télécharger} 
  l'archive contenant MongoDB.
  \paragraph{} Après le téléchargement de l'archive, vous pourrez la décompresser. Avant
  de lancer le serveur, il vous faudra créer le dossier \textbf{/data/db}. Ce dossier
  contiendra toutes les bases de MongoDB.
  \paragraph{} Enfin vous pouvez lancer le serveur MongoDB avec la commande:
  \begin{lstlisting}
  $ ./bin/mongod
  \end{lstlisting}
  Vous pourrez ensuite accéder à MongoDB en lançant le client grâce à la commande suivante:
  \begin{lstlisting}
  $ ./bin/mongo
  \end{lstlisting}
  
  
\subsection{Les méthodes d’insertion de données}
  \paragraph{Création de la base} Avant toute insertion, lancez le client MongoDB
  avec la commande précédente. Nous allons ajouter une nouvelle base de donnée. 
  Vous pouvez lister les différentes base de données déjà présentes avec la commande:
  \begin{lstlisting}
  > show db
  \end{lstlisting}
  Puis pour créer et/ou utiliser notre base de donnée, utilisez:
  \begin{lstlisting}
  > use DATABASE
  \end{lstlisting}
\subsection{Le langage de recherche}
  \paragraph{} TODO
\subsection{L’indexation interne}
  \paragraph{} TODO
\subsection{Le support de la concurrence d’accès}
  \paragraph{} TODO
\subsection{L’architecture du système}
  \paragraph{} TODO
\subsection{Les techniques de distribution}
  \paragraph{} TODO


\section{Conclusion}
  \paragraph{} TODO
  
\end{document}