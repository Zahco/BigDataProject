\documentclass{article}

\usepackage[utf8]{inputenc}
\usepackage[T1]{fontenc}
\usepackage{hyperref}
\usepackage{tabularx}
\usepackage{array}
\usepackage{fancyhdr}
\usepackage{graphicx}
\usepackage[a4paper]{geometry}
\usepackage{multicol}
\usepackage{listings}
\usepackage{tabto}

\title{Projet de BigData \\ Les bases NoSQL}
\author{par Jordan Baudin, Corentin LeGuen et Geoffrey Spaur}
\date{11 janvier 2018}
\pagestyle{fancy}
\lhead{Projet de BigData - Les bases NoSQL \\ \textbf{M2GIL} - Jordan Baudin, Corentin LeGuen et Geoffrey Spaur}
\rhead{\includegraphics[scale=0.5]{logo_univ_rouen.png}}
\setlength{\headsep}{1cm}
\begin{document}

\maketitle
\newpage
\tableofcontents{}
\newpage
\section{Présentation}
  \paragraph{}
  Ce projet a pour but de comparer différentes bases de données NoSQL.
  Nous allons prendre les bases données CouchBase et MongoDB.
  
 \newpage
\section{La base de donnée CouchBase}
\subsection{Les modalités d’installation}
  \paragraph{Recommandations système} \
  \
    \begin{tabular}{|l|l|}
      \hline
      CPU & 6 coeurs 64-bits à 3GHz \\ \hline
      RAM & 16 GB physique          \\ \hline
    \end{tabular}
    TODO
  
  \paragraph{Installation} \
  
  L’installation de CouchBase peut se faire via \href{https://www.couchbase.com/}{le site officiel} avec un .dpkg soit en ligne de commande :\\
  
  \includegraphics[scale=0.8]{couchBase/install.png}\\

  Le service couchbase-server utilise les ports suivants :\\

  \includegraphics[scale=0.8]{couchBase/ports.png}\\
  \newpage
  Le service couchbase-server peut être coupé avec la ligne de code

  \begin{lstlisting}
  sudo service apt-get couchbase-server stop
  \end{lstlisting}

\newpage  
\subsection{Présentation}
  \paragraph{} 
  Le service couchbase est disponible à l’adresse localhost:8091/ui/index.html\\
  
  \includegraphics[scale=0.2]{couchBase/accueil.png}\\
  Par défaut, lors de la première utilisation, le service demande un mot de passe pour l’utilisateur Administrator.\\
 
  \includegraphics[scale=0.2]{couchBase/dashboard.png}\\

  Sur le dashboard, on peut voir les statistiques d’utilisation et d’accès à CouchBase.\

  Avant la toute première utilisation, il est nécessaire de créer un Bucket (qui peut être vu comme un cluster) pour pouvoir insérer et manipuler des données.\\

  \includegraphics[scale=0.2]{couchBase/bucket.png}\\

  Ici, j’ai créé un bucket BigDataMainBucket.\\

  \includegraphics[scale=0.2]{couchBase/document.png}\\

\subsection{Programme Java}
  \paragraph{} 
  Nous avons développé un programme en Java pour insérer des données dans mon bucket. Le code source est disponible \href{https://github.com/CorentinLeGuen/Data-Mining}{ici}.\
  
Une documentation complète et un code riche et détaillé y est présente.
Le programme repose sur un JDBC CouchBase et sur un parseurJSON fait en java :\\

\includegraphics[scale=0.6]{couchBase/java.png}\\

Après l’insertion d’un élément en JSON, grâce à l’AJAX, le résultat est affiché directement (pas besoin de recharger la page). \\

\includegraphics[scale=0.2]{couchBase/ajoutjava.png}\\

\newpage
CouchDB permet de visionner directement les documents enregistrés.\\

\includegraphics[scale=0.2]{couchBase/docjava.png}\\

On peut également réaliser des requêtes N1QL sur CouchDB. \\

\includegraphics[scale=0.2]{couchBase/N1ql.png}\\

Le dashboard permet de visionner les opérations réalisées sur le serveur.\\

\includegraphics[scale=0.2]{couchBase/op.png}

\newpage

\subsection{Performances}
  \paragraph{} 
  Les performances de CouchDB sont disponibles sur le site d'\href{https://www.arangodb.com/2012/08/benchmarking-arangodb-and-couchdb/}{arangoDB}\\

  \includegraphics[scale=0.6]{couchBase/adb.png}\\

  Voici les performances comparé à Persevere et PHP+Mysql :\\

  \includegraphics[scale=0.6]{couchBase/perf.png}\\
  
\newpage
\section{La base de donnée MongoDB}
\subsection{Les modalités d’installation}
  \paragraph{Recommandations système} 
    Il est recommandé par la documentation d'utiliser des processeurs 64-bits car sinon,
    la mémoire sera limité à 2 GB.
    Les performances du serveur sont fortement dépendantes de la RAM du système.
  \paragraph{Téléchargement} Pour l'installation de MongoDB, il vous suffira
  dans un premier temps de \href{https://www.mongodb.com/download-center#community}{télécharger} 
  l'archive contenant MongoDB.
  \paragraph{} Après le téléchargement de l'archive, vous pourrez la décompresser. Avant
  de lancer le serveur, il vous faudra créer le dossier \textbf{/data/db}. Ce dossier
  contiendra toutes les bases de MongoDB.
  \paragraph{} Enfin vous pouvez lancer le serveur MongoDB avec la commande:
  \begin{lstlisting}
  $ ./bin/mongod
  \end{lstlisting}
  \includegraphics[scale=0.8]{mongodb/lancement_mongo.png}\\
  Vous pourrez ensuite accéder à MongoDB en lançant le client grâce à la commande suivante:
  \begin{lstlisting}
  $ ./bin/mongo
  \end{lstlisting}
  
  
\subsection{Les méthodes d’insertion de données}
  \paragraph{Création de la base} Avant toute insertion, lancez le client MongoDB
  avec la commande précédente. Nous allons ajouter une nouvelle base de donnée. 
  Vous pouvez lister les différentes base de données déjà présentes avec la commande:
  \begin{lstlisting}
  > show dbs
  \end{lstlisting}
  \includegraphics[scale=0.8]{mongodb/show_dbs.png}\\
  Puis pour créer et/ou utiliser notre base de donnée, utilisez:
  \begin{lstlisting}
  > use DATABASE
  \end{lstlisting}
  \includegraphics[scale=0.8]{mongodb/use_dbname.png}\\
  Vous pouvez lister vos collections de document avec la commande suivante:
  \begin{lstlisting}
  > show collections
  \end{lstlisting}
  \includegraphics[scale=0.8]{mongodb/show_collections.png}\\
  
  \paragraph{Insertion de documents}
  Vous pouvez maintenant sortir du client MongoDB. Vos documents doivent être en
  format json sous forme d'objet. Vous trouverez un exemple avec le fichier 
  \emph{prettyAlcoholism.json}. Nous allons utiliser la commande
  \emph{mongoimport} afin d'ajouter nos document dans notre base:
  \begin{lstlisting}
  $ ./bin/mongoimport --db bigdata --collection medarticles 
  --file ../Projet/prettyAlcoholism.json
  \end{lstlisting}
  \includegraphics[scale=0.8]{mongodb/insertiondedoc2.png}\\
  \paragraph{} Vous pouvez cependant ajouter des document directement à partir
  du client MongoDB avec les commandes \emph{import} ou \emph{importMany}.
  Ces commandes sont utiles dans le cas où nous avons peu de document ou des
  document de petites tailles.
  
\subsection{Le langage de recherche}
  \paragraph{} Pour rechercher des documents dans MongoDB, nous utiliserons la 
  fonction \emph{find()} sur nos collections. Cette commande peut prendre jusqu'à
  deux arguments sous format JSON:
  \begin{itemize}
    \item Le premier permet de déterminer la clause \textbf{where}.
    \item La seconde permet de spécifier les champs à afficher.
  \end{itemize}
  \subsubsection{Les conditions}
    \paragraph{} Vous pouvez effectuer une recherche en indiquant la valeur d'un
    champs comme ceci:\\
    \includegraphics[scale=0.8]{mongodb/recherche_find1.png}\\
    En ajoutant la fonction \emph{pretty()}, le résultat sera indenté.
    Vous pouvez utiliser les tableaux \emph{\$and} et \emph{\$or} afin de constituer
    des recherches plus complexes.
  \subsubsection{L'affichage}
    \paragraph{} Vous pouvez aussi n'afficher que certain champs dans le résultat
    de votre recherche:\\
    \includegraphics[scale=0.8]{mongodb/recherche_find2.png}\\
  
\subsection{L’indexation interne}
  \paragraph{} Pour chaque document ajouter dans une collection, MongoDB lui ajoute
  un champs \emph{\_id}. C'est ce champs qui sera indexé par défaut. Il est 
  possible de lister tous les indexes d'une collection avec la commande suivant:
  \begin{lstlisting}
  > db.medarticles.getIndexes()
  \end{lstlisting}
  \includegraphics[scale=0.8]{mongodb/getindexes.png}\\
  
  \paragraph{} Vous pouvez cependant ajouter vos propres indexes avec la commande
  \emph{ensureIndex()}. L'objet JSON passé en argument doit être une liste de \textbf{int}.
  Le nom doit être le champs à indexer et le \textbf{int} doit être égal à 1 ou -1; 
  la valeur 1 correspond à un tri ascendant et la valeur -1 à un tri descendant.
  \includegraphics[scale=0.8]{mongodb/ensureIndex.png}\\
  
  \paragraph{} Afin de mesurer l'impact de vos indexes, à chaque recherche vous pouvez
  ajouter la méthode \emph{explain}:\\
  \includegraphics[scale=0.8]{mongodb/find_explain2.png}\\
\subsection{Le support de la concurrence d’accès}
  \paragraph{} MongoDB support totalement la concurrence d'accès.
  Il permet à plusieurs clients de lire et écrire les mêmes données.
  MongoDB gère la concurrence à l'aide de verrous. Certaines opérations entraînent
  un verrouillage de la collection ou même de la base de données. Par exemple
  les opérations d'insertion, de suppression, de modification ou encore de
  création d'index entraînent un verrouillage de la base de données.
\subsection{L’architecture du système}
  \includegraphics[scale=0.215]{mongodb/mongodb-arch-diag.png}\\
\subsection{Les techniques de distribution}
  \includegraphics[scale=0.7]{mongodb/sharded-mongodb.png}\\

\newpage
\section{Conclusion}
  \paragraph{}
  CouchDB et MongoDB sont des bases de données avec des objectifs différents,
  MongoDB sera utilisé pour interagir avec une application lié à un JDBC alors que CouchDB sera plus largement utilisé
  comme une interface REST, grâce à sa construction basé sur du HTTP.\
  
  MongoDB est plus rapide que CouchDB, mais ne permet pas son utilisation sur téléphone.\
  Une base MongoDB pourra être agrandi avec facilité là où CouchDB aura plus de difficulté, mais CouchDB reste le meilleur choix pour une utilisation mobile, avec une réplication maître à maître ou un unique serveur.
  
\end{document}